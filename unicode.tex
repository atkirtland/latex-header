% borrowed lots from
% https://github.com/mbrockman1/MikeMacs/blob/0a01ad8bafc64d11d3a1c65a702641f5da5c4876/elpa/latex-unicode-math-mode-20170123.1016/unicode-math-mode.sty

% greek letters

\DeclareUnicodeCharacter{03B1}{\ensuremath{\alpha}}
\DeclareUnicodeCharacter{03B9}{\ensuremath{\iota}}
\DeclareUnicodeCharacter{03B2}{\ensuremath{\beta}}
\DeclareUnicodeCharacter{03BA}{\ensuremath{\kappa}}
\DeclareUnicodeCharacter{03F0}{\ensuremath{\varkappa}}
\DeclareUnicodeCharacter{03C3}{\ensuremath{\sigma}}
\DeclareUnicodeCharacter{03B3}{\ensuremath{\gamma}}
\DeclareUnicodeCharacter{03BB}{\ensuremath{\lambda}}
\DeclareUnicodeCharacter{03B4}{\ensuremath{\delta}}
\DeclareUnicodeCharacter{03BC}{\ensuremath{\mu}} %! mü, wird in Neo nicht verwendet
\DeclareUnicodeCharacter{00B5}{\ensuremath{\mu}} %! micro
\DeclareUnicodeCharacter{03C4}{\ensuremath{\tau}}
\DeclareUnicodeCharacter{03BD}{\ensuremath{\nu}}
\DeclareUnicodeCharacter{03C5}{\ensuremath{\upsilon}}
\DeclareUnicodeCharacter{03F5}{\ensuremath{\epsilon}}
\DeclareUnicodeCharacter{03B5}{\ensuremath{\varepsilon}}
\DeclareUnicodeCharacter{03BE}{\ensuremath{\xi}}
\DeclareUnicodeCharacter{03B6}{\ensuremath{\zeta}}
\DeclareUnicodeCharacter{03D5}{\ensuremath{\phi}}
\DeclareUnicodeCharacter{03C6}{\ensuremath{\varphi}}
\DeclareUnicodeCharacter{03B7}{\ensuremath{\eta}}
\DeclareUnicodeCharacter{03C0}{\ensuremath{\pi}}
\DeclareUnicodeCharacter{03D6}{\ensuremath{\varpi}}
\DeclareUnicodeCharacter{03C7}{\ensuremath{\chi}}
\DeclareUnicodeCharacter{03B8}{\ensuremath{\theta}}
\DeclareUnicodeCharacter{03C8}{\ensuremath{\psi}}
\DeclareUnicodeCharacter{03D1}{\ensuremath{\vartheta}}
\DeclareUnicodeCharacter{03C1}{\ensuremath{\rho}}
\DeclareUnicodeCharacter{03F1}{\ensuremath{\varrho}}
\DeclareUnicodeCharacter{03C9}{\ensuremath{\omega}}
\DeclareUnicodeCharacter{0393}{\ensuremath{\Gamma}}
\DeclareUnicodeCharacter{039E}{\ensuremath{\Xi}}
\DeclareUnicodeCharacter{03A6}{\ensuremath{\Phi}}
\DeclareUnicodeCharacter{0394}{\ensuremath{\Delta}}
\DeclareUnicodeCharacter{03A0}{\ensuremath{\Pi}}
\DeclareUnicodeCharacter{03A8}{\ensuremath{\Psi}}
\DeclareUnicodeCharacter{0398}{\ensuremath{\Theta}}
\DeclareUnicodeCharacter{03A3}{\ensuremath{\Sigma}}
\DeclareUnicodeCharacter{03A9}{\ensuremath{\Omega}}
\DeclareUnicodeCharacter{039B}{\ensuremath{\Lambda}}
\DeclareUnicodeCharacter{03A7}{\ensuremath{\Chi}}

% calculus

\DeclareUnicodeCharacter{2207}{\ensuremath{\nabla}}
\DeclareUnicodeCharacter{2202}{\ensuremath{\partial}}
\DeclareUnicodeCharacter{222C}{\ensuremath{\iint}}
\DeclareUnicodeCharacter{222D}{\ensuremath{\iiint}}
\DeclareUnicodeCharacter{2A0C}{\ensuremath{\iiiint}}
\DeclareUnicodeCharacter{222E}{\ensuremath{\oint}}
\DeclareUnicodeCharacter{222F}{\ensuremath{\oiint}}
\DeclareUnicodeCharacter{2230}{\ensuremath{\oiiint}}

% logic

\DeclareUnicodeCharacter{2254}{:=}
\DeclareUnicodeCharacter{2255}{=:}
\DeclareUnicodeCharacter{2203}{\ensuremath{\exists}}
\DeclareUnicodeCharacter{2200}{\ensuremath{\forall}}

% basic

\DeclareUnicodeCharacter{22A4}{\top}                    %         ⊤
\DeclareUnicodeCharacter{22A5}{\bot}                    %         ⊥
\DeclareUnicodeCharacter{2135}{\ensuremath{\aleph}}
\DeclareUnicodeCharacter{2211}{\ensuremath{\sum}}
\DeclareUnicodeCharacter{222B}{\ensuremath{\int}}
\DeclareUnicodeCharacter{220F}{\ensuremath{\prod}}
\DeclareUnicodeCharacter{2261}{\ensuremath{\equiv}}
\DeclareUnicodeCharacter{2020}{\ensuremath{\dagger}} % †
\DeclareUnicodeCharacter{00AC}{\ensuremath{\neg}}
\DeclareUnicodeCharacter{33D1}{\ensuremath{\ln}}
\DeclareUnicodeCharacter{33D2}{\ensuremath{\log}}
\DeclareUnicodeCharacter{221D}{\propto} % ∝
\DeclareUnicodeCharacter{211C}{\ensuremath{\Re}}
\DeclareUnicodeCharacter{2111}{\ensuremath{\Im}}
\DeclareUnicodeCharacter{00A0}{~}
\DeclareUnicodeCharacter{202F}{\,}
\DeclareUnicodeCharacter{230A}{\ensuremath{\lfloor}}
\DeclareUnicodeCharacter{230B}{\ensuremath{\rfloor}}
\DeclareUnicodeCharacter{221A}{\ensuremath{\sqrt}}
\DeclareUnicodeCharacter{221B}{\ensuremath{\sqrt[3]}}
\DeclareUnicodeCharacter{221C}{\ensuremath{\sqrt[4]}}
\DeclareUnicodeCharacter{00D7}{\ensuremath{\times}}
\DeclareUnicodeCharacter{00F7}{\ensuremath{\div}}
\DeclareUnicodeCharacter{00B1}{\ensuremath{\pm}}
\DeclareUnicodeCharacter{2213}{\ensuremath{\mp}}
\DeclareUnicodeCharacter{2215}{\ensuremath{/}}
\DeclareUnicodeCharacter{2212}{\ensuremath{-}}
\DeclareUnicodeCharacter{221E}{\ensuremath{\infty}}
\DeclareUnicodeCharacter{2228}{\ensuremath{\vee}}
\DeclareUnicodeCharacter{2227}{\ensuremath{\wedge}}

% big center

\DeclareUnicodeCharacter{22C1}{\ensuremath{\bigvee}}
\DeclareUnicodeCharacter{22C0}{\ensuremath{\bigwedge}}
\DeclareUnicodeCharacter{22C3}{\ensuremath{\bigcup}}
\DeclareUnicodeCharacter{22C2}{\ensuremath{\bigcap}}
\DeclareUnicodeCharacter{2A00}{\ensuremath{\bigodot}}
\DeclareUnicodeCharacter{2A01}{\ensuremath{\bigoplus}}
\DeclareUnicodeCharacter{2A02}{\ensuremath{\bigotimes}}

% orders

\DeclareUnicodeCharacter{2260}{\ensuremath{\neq}}
\DeclareUnicodeCharacter{2248}{\ensuremath{\approx}}
\DeclareUnicodeCharacter{2264}{\ensuremath{\leq}}
\DeclareUnicodeCharacter{2265}{\ensuremath{\geq}}
\DeclareUnicodeCharacter{226A}{\ensuremath{\ll}}
\DeclareUnicodeCharacter{226B}{\ensuremath{\gg}}
\DeclareUnicodeCharacter{226D}{\not\equiv}              %         ≉
\DeclareUnicodeCharacter{2270}{\not\le}                 %         ≱ 
\DeclareUnicodeCharacter{2271}{\not\ge}                 %         ≱ 
\DeclareUnicodeCharacter{2272}{\lesssim}                %         ≲
\DeclareUnicodeCharacter{2273}{\gtrsim}                 %         ≳
\DeclareUnicodeCharacter{227A}{\prec}                   %         ≺
\DeclareUnicodeCharacter{2282}{\subset}                 %         ⊂
\DeclareUnicodeCharacter{2286}{\subseteq}               %         ⊆
\DeclareUnicodeCharacter{228A}{\subsetneq}               %         ⊊


% sets

\DeclareUnicodeCharacter{220B}{\ensuremath{\ni}}
\DeclareUnicodeCharacter{2205}{\ensuremath{\emptyset}}
\DeclareUnicodeCharacter{2208}{\ensuremath{\in}}
\DeclareUnicodeCharacter{2282}{\ensuremath{\subset}}
\DeclareUnicodeCharacter{2283}{\ensuremath{\supset}}
\DeclareUnicodeCharacter{2286}{\ensuremath{\subseteq}}
\DeclareUnicodeCharacter{2287}{\ensuremath{\supseteq}}
\DeclareUnicodeCharacter{2229}{\ensuremath{\cap}}
\DeclareUnicodeCharacter{222A}{\ensuremath{\cup}}
\DeclareUnicodeCharacter{2288}{\ensuremath{\nsubseteq}} %! ist nur per Compose zu erreichen

% dots

\DeclareUnicodeCharacter{2026}{\ifmmode\ldots\else\textellipsis\fi} % nutze den jeweils passenden Befehl
\DeclareUnicodeCharacter{22C5}{\ensuremath{\cdot}}
\DeclareUnicodeCharacter{22EE}{\ensuremath{\vdots}}                  %         ⋮

% arrows

\DeclareUnicodeCharacter{21D2}{\ensuremath{\Rightarrow}}
\DeclareUnicodeCharacter{21D0}{\ensuremath{\Leftarrow}}
\DeclareUnicodeCharacter{21D4}{\ensuremath{\Leftrightarrow}}
\DeclareUnicodeCharacter{2192}{\ensuremath{\to}}
\DeclareUnicodeCharacter{2190}{\ensuremath{\gets}}
\DeclareUnicodeCharacter{21A6}{\ensuremath{\mapsto}}

% misc

\DeclareUnicodeCharacter{20AC}{\EUR}

% exponent of numbers

\DeclareUnicodeCharacter{2070}{\ensuremath{^0}}
\DeclareUnicodeCharacter{00B9}{\ensuremath{^1}}
\DeclareUnicodeCharacter{00B2}{\ensuremath{^2}}
\DeclareUnicodeCharacter{00B3}{\ensuremath{^3}}
\DeclareUnicodeCharacter{2074}{\ensuremath{^4}}
\DeclareUnicodeCharacter{2075}{\ensuremath{^5}}
\DeclareUnicodeCharacter{2076}{\ensuremath{^6}}
\DeclareUnicodeCharacter{2077}{\ensuremath{^7}}
\DeclareUnicodeCharacter{2078}{\ensuremath{^8}}
\DeclareUnicodeCharacter{2079}{\ensuremath{^9}}
\DeclareUnicodeCharacter{207A}{\ensuremath{^+}}
\DeclareUnicodeCharacter{207B}{\ensuremath{^-}}
\DeclareUnicodeCharacter{207C}{\ensuremath{^=}}
\DeclareUnicodeCharacter{207D}{\ensuremath{^(}}
\DeclareUnicodeCharacter{207E}{\ensuremath{^)}}
\DeclareUnicodeCharacter{2080}{\ensuremath{_0}}
\DeclareUnicodeCharacter{2081}{\ensuremath{_1}}
\DeclareUnicodeCharacter{2082}{\ensuremath{_2}}
\DeclareUnicodeCharacter{2083}{\ensuremath{_3}}
\DeclareUnicodeCharacter{2084}{\ensuremath{_4}}
\DeclareUnicodeCharacter{2085}{\ensuremath{_5}}
\DeclareUnicodeCharacter{2086}{\ensuremath{_6}}
\DeclareUnicodeCharacter{2087}{\ensuremath{_7}}
\DeclareUnicodeCharacter{2088}{\ensuremath{_8}}
\DeclareUnicodeCharacter{2089}{\ensuremath{_9}}
\DeclareUnicodeCharacter{208A}{\ensuremath{_+}}
\DeclareUnicodeCharacter{208B}{\ensuremath{_-}}
\DeclareUnicodeCharacter{208C}{\ensuremath{_=}}
\DeclareUnicodeCharacter{208D}{\ensuremath{_(}}
\DeclareUnicodeCharacter{208E}{\ensuremath{_)}}

%  Latin Subscript Small Letters - i got the first few from a repo and the rest of the alphabet doesn't seem to follow an easy pattern (note u)

\DeclareUnicodeCharacter{2096}{_k}
\DeclareUnicodeCharacter{2097}{_l}
\DeclareUnicodeCharacter{2098}{_m}
\DeclareUnicodeCharacter{2099}{_n}
\DeclareUnicodeCharacter{2099}{_n}
\DeclareUnicodeCharacter{209B}{_s}
\DeclareUnicodeCharacter{209C}{_t}
\DeclareUnicodeCharacter{1D64}{_u}

% \mathbb

\DeclareUnicodeCharacter{1D538}{\ensuremath{\mathbb{A}}}
\DeclareUnicodeCharacter{1D539}{\ensuremath{\mathbb{B}}}
\DeclareUnicodeCharacter{02102}{\ensuremath{\mathbb{C}}}
\DeclareUnicodeCharacter{1D53B}{\ensuremath{\mathbb{D}}}
\DeclareUnicodeCharacter{1D53C}{\ensuremath{\mathbb{E}}}
\DeclareUnicodeCharacter{1D53D}{\ensuremath{\mathbb{F}}}
\DeclareUnicodeCharacter{1D53E}{\ensuremath{\mathbb{G}}}
\DeclareUnicodeCharacter{0210D}{\ensuremath{\mathbb{H}}}
\DeclareUnicodeCharacter{1D540}{\ensuremath{\mathbb{I}}}
\DeclareUnicodeCharacter{1D541}{\ensuremath{\mathbb{J}}}
\DeclareUnicodeCharacter{1D542}{\ensuremath{\mathbb{K}}}
\DeclareUnicodeCharacter{1D543}{\ensuremath{\mathbb{L}}}
\DeclareUnicodeCharacter{1D544}{\ensuremath{\mathbb{M}}}
\DeclareUnicodeCharacter{02115}{\ensuremath{\mathbb{N}}}
\DeclareUnicodeCharacter{1D546}{\ensuremath{\mathbb{O}}}
\DeclareUnicodeCharacter{02119}{\ensuremath{\mathbb{P}}}
\DeclareUnicodeCharacter{0211A}{\ensuremath{\mathbb{Q}}}
\DeclareUnicodeCharacter{0211D}{\ensuremath{\mathbb{R}}}
\DeclareUnicodeCharacter{1D54A}{\ensuremath{\mathbb{S}}}
\DeclareUnicodeCharacter{1D54B}{\ensuremath{\mathbb{T}}}
\DeclareUnicodeCharacter{1D54C}{\ensuremath{\mathbb{U}}}
\DeclareUnicodeCharacter{1D54D}{\ensuremath{\mathbb{V}}}
\DeclareUnicodeCharacter{1D54E}{\ensuremath{\mathbb{W}}}
\DeclareUnicodeCharacter{1D54F}{\ensuremath{\mathbb{X}}}
\DeclareUnicodeCharacter{1D550}{\ensuremath{\mathbb{Y}}}
\DeclareUnicodeCharacter{02124}{\ensuremath{\mathbb{Z}}}
\DeclareUnicodeCharacter{1D552}{\ensuremath{\mathbb{a}}}
\DeclareUnicodeCharacter{1D553}{\ensuremath{\mathbb{b}}}
\DeclareUnicodeCharacter{1D554}{\ensuremath{\mathbb{c}}}
\DeclareUnicodeCharacter{1D555}{\ensuremath{\mathbb{d}}}
\DeclareUnicodeCharacter{1D556}{\ensuremath{\mathbb{e}}}
\DeclareUnicodeCharacter{1D557}{\ensuremath{\mathbb{f}}}
\DeclareUnicodeCharacter{1D558}{\ensuremath{\mathbb{g}}}
\DeclareUnicodeCharacter{1D559}{\ensuremath{\mathbb{h}}}
\DeclareUnicodeCharacter{1D55A}{\ensuremath{\mathbb{i}}}
\DeclareUnicodeCharacter{1D55B}{\ensuremath{\mathbb{j}}}
\DeclareUnicodeCharacter{1D55C}{\ensuremath{\mathbb{k}}}
\DeclareUnicodeCharacter{1D55D}{\ensuremath{\mathbb{l}}}
\DeclareUnicodeCharacter{1D55E}{\ensuremath{\mathbb{m}}}
\DeclareUnicodeCharacter{1D55F}{\ensuremath{\mathbb{n}}}
\DeclareUnicodeCharacter{1D560}{\ensuremath{\mathbb{o}}}
\DeclareUnicodeCharacter{1D561}{\ensuremath{\mathbb{p}}}
\DeclareUnicodeCharacter{1D562}{\ensuremath{\mathbb{q}}}
\DeclareUnicodeCharacter{1D563}{\ensuremath{\mathbb{r}}}
\DeclareUnicodeCharacter{1D564}{\ensuremath{\mathbb{s}}}
\DeclareUnicodeCharacter{1D565}{\ensuremath{\mathbb{t}}}
\DeclareUnicodeCharacter{1D566}{\ensuremath{\mathbb{u}}}
\DeclareUnicodeCharacter{1D567}{\ensuremath{\mathbb{v}}}
\DeclareUnicodeCharacter{1D568}{\ensuremath{\mathbb{w}}}
\DeclareUnicodeCharacter{1D569}{\ensuremath{\mathbb{x}}}
\DeclareUnicodeCharacter{1D56A}{\ensuremath{\mathbb{y}}}
\DeclareUnicodeCharacter{1D56B}{\ensuremath{\mathbb{z}}}
\DeclareUnicodeCharacter{1D7D8}{\ensuremath{\mathbb{0}}}
\DeclareUnicodeCharacter{1D7D9}{\ensuremath{\mathbb{1}}}
\DeclareUnicodeCharacter{1D7DA}{\ensuremath{\mathbb{2}}}
\DeclareUnicodeCharacter{1D7DB}{\ensuremath{\mathbb{3}}}
\DeclareUnicodeCharacter{1D7DC}{\ensuremath{\mathbb{4}}}
\DeclareUnicodeCharacter{1D7DD}{\ensuremath{\mathbb{5}}}
\DeclareUnicodeCharacter{1D7DE}{\ensuremath{\mathbb{6}}}
\DeclareUnicodeCharacter{1D7DF}{\ensuremath{\mathbb{7}}}
\DeclareUnicodeCharacter{1D7E0}{\ensuremath{\mathbb{8}}}
\DeclareUnicodeCharacter{1D7E1}{\ensuremath{\mathbb{9}}}

% \mathfrak

\DeclareUnicodeCharacter{1D504}{\ensuremath{\mathfrak{A}}}
\DeclareUnicodeCharacter{1D505}{\ensuremath{\mathfrak{B}}}
\DeclareUnicodeCharacter{1D507}{\ensuremath{\mathfrak{D}}}
\DeclareUnicodeCharacter{1D508}{\ensuremath{\mathfrak{E}}}
\DeclareUnicodeCharacter{1D509}{\ensuremath{\mathfrak{F}}}
\DeclareUnicodeCharacter{1D50A}{\ensuremath{\mathfrak{G}}}
\DeclareUnicodeCharacter{1D50D}{\ensuremath{\mathfrak{J}}}
\DeclareUnicodeCharacter{1D50E}{\ensuremath{\mathfrak{K}}}
\DeclareUnicodeCharacter{1D50F}{\ensuremath{\mathfrak{L}}}
\DeclareUnicodeCharacter{1D510}{\ensuremath{\mathfrak{M}}}
\DeclareUnicodeCharacter{1D511}{\ensuremath{\mathfrak{N}}}
\DeclareUnicodeCharacter{1D512}{\ensuremath{\mathfrak{O}}}
\DeclareUnicodeCharacter{1D513}{\ensuremath{\mathfrak{P}}}
\DeclareUnicodeCharacter{1D514}{\ensuremath{\mathfrak{Q}}}
\DeclareUnicodeCharacter{1D516}{\ensuremath{\mathfrak{S}}}
\DeclareUnicodeCharacter{1D517}{\ensuremath{\mathfrak{T}}}
\DeclareUnicodeCharacter{1D518}{\ensuremath{\mathfrak{U}}}
\DeclareUnicodeCharacter{1D519}{\ensuremath{\mathfrak{V}}}
\DeclareUnicodeCharacter{1D51A}{\ensuremath{\mathfrak{W}}}
\DeclareUnicodeCharacter{1D51B}{\ensuremath{\mathfrak{X}}}
\DeclareUnicodeCharacter{1D51C}{\ensuremath{\mathfrak{Y}}}

\DeclareUnicodeCharacter{1D524}{\ensuremath{\mathfrak{g}}}
\DeclareUnicodeCharacter{1D529}{\ensuremath{\mathfrak{l}}}

% \mathcal

\DeclareUnicodeCharacter{1D4D0}{\ensuremath{\mathcal{A}}}
\DeclareUnicodeCharacter{1D4D1}{\ensuremath{\mathcal{B}}}
\DeclareUnicodeCharacter{1D4D2}{\ensuremath{\mathcal{C}}}
\DeclareUnicodeCharacter{1D4D3}{\ensuremath{\mathcal{D}}}
\DeclareUnicodeCharacter{1D4D4}{\ensuremath{\mathcal{E}}}
\DeclareUnicodeCharacter{1D4D5}{\ensuremath{\mathcal{F}}}
\DeclareUnicodeCharacter{1D4D6}{\ensuremath{\mathcal{G}}}
\DeclareUnicodeCharacter{1D4D7}{\ensuremath{\mathcal{H}}}
\DeclareUnicodeCharacter{1D4D8}{\ensuremath{\mathcal{I}}}
\DeclareUnicodeCharacter{1D4D9}{\ensuremath{\mathcal{J}}}
\DeclareUnicodeCharacter{1D4DA}{\ensuremath{\mathcal{K}}}
\DeclareUnicodeCharacter{1D4DB}{\ensuremath{\mathcal{L}}}
\DeclareUnicodeCharacter{1D4DC}{\ensuremath{\mathcal{M}}}
\DeclareUnicodeCharacter{1D4DD}{\ensuremath{\mathcal{N}}}
\DeclareUnicodeCharacter{1D4DE}{\ensuremath{\mathcal{O}}}
\DeclareUnicodeCharacter{1D4DF}{\ensuremath{\mathcal{P}}}
\DeclareUnicodeCharacter{1D4E0}{\ensuremath{\mathcal{Q}}}
\DeclareUnicodeCharacter{1D4E1}{\ensuremath{\mathcal{R}}}
\DeclareUnicodeCharacter{1D4E2}{\ensuremath{\mathcal{S}}}
\DeclareUnicodeCharacter{1D4E3}{\ensuremath{\mathcal{T}}}
\DeclareUnicodeCharacter{1D4E4}{\ensuremath{\mathcal{U}}}
\DeclareUnicodeCharacter{1D4E5}{\ensuremath{\mathcal{V}}}
\DeclareUnicodeCharacter{1D4E6}{\ensuremath{\mathcal{W}}}
\DeclareUnicodeCharacter{1D4E7}{\ensuremath{\mathcal{X}}}
\DeclareUnicodeCharacter{1D4E8}{\ensuremath{\mathcal{Y}}}
\DeclareUnicodeCharacter{1D4E9}{\ensuremath{\mathcal{Z}}}

% exponent letters

\DeclareUnicodeCharacter{1D43}{^a}
\DeclareUnicodeCharacter{1D47}{^b}
\DeclareUnicodeCharacter{1D9C}{^c}
\DeclareUnicodeCharacter{1D48}{^d}
\DeclareUnicodeCharacter{1D49}{^e}
\DeclareUnicodeCharacter{1DA0}{^f}
\DeclareUnicodeCharacter{1D4D}{^g}
\DeclareUnicodeCharacter{02B0}{^h}
\DeclareUnicodeCharacter{2071}{^i}
\DeclareUnicodeCharacter{02B2}{^j}
\DeclareUnicodeCharacter{1D4F}{^k}
\DeclareUnicodeCharacter{02E1}{^l}
\DeclareUnicodeCharacter{1D50}{^m}
\DeclareUnicodeCharacter{207F}{^n}
\DeclareUnicodeCharacter{1D52}{^o}
\DeclareUnicodeCharacter{1D56}{^p}
\DeclareUnicodeCharacter{02B3}{^r}
\DeclareUnicodeCharacter{02E2}{^s}
\DeclareUnicodeCharacter{1D57}{^t}
\DeclareUnicodeCharacter{1D58}{^u}
\DeclareUnicodeCharacter{1D5B}{^v}
\DeclareUnicodeCharacter{02B7}{^w}
\DeclareUnicodeCharacter{02E3}{^x}
\DeclareUnicodeCharacter{02B8}{^y}
\DeclareUnicodeCharacter{1DBB}{^z}
\DeclareUnicodeCharacter{1D2C}{^A}
\DeclareUnicodeCharacter{1D2E}{^B}
\DeclareUnicodeCharacter{1D30}{^D}
\DeclareUnicodeCharacter{1D31}{^E}
\DeclareUnicodeCharacter{1D33}{^G}
\DeclareUnicodeCharacter{1D34}{^H}
\DeclareUnicodeCharacter{1D35}{^I}
\DeclareUnicodeCharacter{1D36}{^J}
\DeclareUnicodeCharacter{1D37}{^K}
\DeclareUnicodeCharacter{1D38}{^L}
\DeclareUnicodeCharacter{1D39}{^M}
\DeclareUnicodeCharacter{1D3A}{^N}
\DeclareUnicodeCharacter{1D3C}{^O}
\DeclareUnicodeCharacter{1D3E}{^P}
\DeclareUnicodeCharacter{1D3F}{^R}
\DeclareUnicodeCharacter{1D40}{^T}
\DeclareUnicodeCharacter{1D41}{^U}
\DeclareUnicodeCharacter{1D42}{^W}

% \mathcal, haven't tried to add my own yet to complete the alphabet

\DeclareUnicodeCharacter{1D49E}{\mathcal{C}}            %    𝒞
\DeclareUnicodeCharacter{1D49F}{\mathcal{D}}            %    𝒟
\DeclareUnicodeCharacter{1D4BB}{\mathcal{f}}            %    𝒻
\DeclareUnicodeCharacter{1D4DB}{\mathcal{L}}            %    𝓛
\DeclareUnicodeCharacter{1D4E3}{\mathcal{T}}            %    𝓣

% NEW

\DeclareUnicodeCharacter{2308}{\left\lceil}
\DeclareUnicodeCharacter{2309}{\right\rceil}
\DeclareUnicodeCharacter{1D62}{\ensuremath{_i}}
\DeclareUnicodeCharacter{00B7}{\ensuremath{\cdot}}
\DeclareUnicodeCharacter{223C}{\ensuremath{\sim}}
\DeclareUnicodeCharacter{00D5}{\ifmmode\tilde{O}\else\~{O}\fi}
\DeclareUnicodeCharacter{2245}{\ensuremath{\cong}}
\DeclareUnicodeCharacter{22EF}{\ensuremath{\cdots}}
\DeclareUnicodeCharacter{2C7C}{\ensuremath{_j}}
\DeclareUnicodeCharacter{209A}{\ensuremath{_p}}
\DeclareUnicodeCharacter{2297}{\ensuremath{\otimes}}
\DeclareUnicodeCharacter{2295}{\oplus}
\DeclareUnicodeCharacter{21AA}{\xhookrightarrow{}}
\DeclareUnicodeCharacter{2016}{\Vert}
\DeclareUnicodeCharacter{2113}{\ell}
\DeclareUnicodeCharacter{2218}{\circ}
\DeclareUnicodeCharacter{2209}{\not\in}
\DeclareUnicodeCharacter{22F1}{\ddots}
\DeclareUnicodeCharacter{22F0}{\udots}

% https://tex.stackexchange.com/questions/26637/how-do-you-get-mathbb1-to-work-characteristic-function-of-a-set
\DeclareUnicodeCharacter{1D7D9}{\mathbbm{1}}