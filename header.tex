\documentclass[11pt,letterpaper]{article}
\usepackage{fullpage}
\usepackage{graphicx,epsfig}
% is this package for pseudocode comments?
\usepackage{comment}
\usepackage{makeidx}

\usepackage{amsmath}
\usepackage{amsfonts} %used for \mathbb{N}
\usepackage{amsthm} % used for Defn
\usepackage{amssymb}

\usepackage[section]{algorithm}
\usepackage{algpseudocode}

\usepackage{tikz}
\usetikzlibrary{trees}

\makeindex

% subsections are listed with letters
\renewcommand\thesubsection{\Alph{subsection}}

\oddsidemargin    0in
\setlength{\evensidemargin}{0in}
\textwidth        6.5in
\topmargin        0.0in
\textheight       8.5in

% borrowed some ideas from https://github.com/NivenT/latex_header

\theoremstyle{plain}
% counter resets at section - I think this is also the default behavior
\newtheorem{thm}{Theorem}[section]
\newtheorem*{prob}{Problem}

\theoremstyle{definition}
% counter shared with thm
\newtheorem{defn}[thm]{Definition}
\newtheorem*{defns}{Definition}

\theoremstyle{remark}
\newtheorem{rem}{Remark}
\newtheorem{eg}[thm]{Example}
\newtheorem{notn}[thm]{Notation}

\newcommand{\define}[1]{\textbf{#1}\index{#1}}

% Letters/Font
\DeclareMathOperator{\F}{\mathbb{F}}
\DeclareMathOperator{\Q}{\mathbb{Q}}
\DeclareMathOperator{\Z}{\mathbb{Z}}
\DeclareMathOperator{\R}{\mathbb{R}}
\DeclareMathOperator{\C}{\mathbb{C}}
\DeclareMathOperator{\E}{\mathbb{E}}
\DeclareMathOperator{\N}{\mathbb{N}}

% Grouping Operators
\newcommand{\floor}[1]{\left\lfloor#1\right\rfloor}
\newcommand{\ceil}[1]{\left\lceil#1\right\rceil}
\newcommand{\parens}[1]{\left(#1\right)}
% note that we defined bracks not brace/braces because brace is already used by amsmath
\newcommand{\bracks}[1]{\left\{#1\right\}}
\newcommand{\sqbracks}[1]{\left[#1\right]}
\newcommand{\clop}[1]{\left[#1\right)}
\newcommand{\opcl}[1]{\left(#1\right]}

\DeclareMathOperator{\Cov}{Cov}
\DeclareMathOperator{\Var}{Var}

\graphicspath{{.}}
